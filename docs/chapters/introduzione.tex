% !TEX encoding = UTF-8
% !TEX TS-program = pdflatex
% !TEX root = ../main.tex

\chapter{Introduzione}

\section{Finalità e ragioni della ricerca}

Il progetto discusso nella presente relazione concerne il riconoscimento di moto implicito e esplicito per mezzo di una rete neurale convoluzionale. Risulta inoltre d'interesse verificare se, e possibilmente come, un determinato strato di una rete neurale si specializzi nel riconoscimento dell'entità di enteresse (nella fattispecie, il moto). Il punto di riferimento iniziale è stata la replica mediante software di un esperimento svolto su volontari umani nell'ambito della ricerca di una correlazione tra una determinata area del cervello e il riconoscimento di certe tipologie di moto. L'esperimento è stato riprodotto il quanto più fedelmente possibile nell'ottica di evidenziare possibili analogie e differenze. I risultati e gli approfondimenti degli esperiementi svolti sono riportati in questo documento.

\section{Definizioni di moto implicito e esplicito}

Nella nozione generale di moto si distinguono due definizioni alla base degli esperimenti condotti e di seguito discussi:
\begin{itemize}
	\item \textbf{Moto implicito}: il moto implicato dalle circostanze di una scena statica (e.g. la fotografia di un velocista nel pieno di una gara);
	\item \textbf{Moto esplicito}: l'effettivo atto del muoversi.
\end{itemize}

\subsection{Glass pattern}

I \textit{glass pattern} sono motivi generati dalla sovrapposizione di due pattern di punti casuali: un modello originale e un secondo generato attraverso una trasformazione lineare o non lineare dell'originale. Sebbene ogni insieme sia casuale, è possibile generare una varietà di diversi modelli spaziali come cerchi, spirali, iperboli, introducendo correlazioni tra i due gruppi di punti. A tal proposito, il moto implicito può essere reso attraverso una sequenza disordinata di punti, mentre quello esplicito inserendo un certo grado. \\

[INSERIRE ESEMPI DI GLASS PATTERN] 

\section{Esperimento di riferimento}

L'esperimento di riferimento per il progetto, condotto su volontari umani, ha l'obiettivo di evidenziare come una certa area cerebrale sia fortemente coinvolta nel riconoscimento di scene di moto implicito e/o esplicito. Tale esperimento espone il volontario alla visione di \textit{glass pattern} appositamente calibrati per rendere più o meno palese la presenza di una determinata tipologia di moto. Il risultato è auspicabilmente raggiunto misurando l'accuratezza di riconoscimento di un dato movimento e confrontando poi i risultati con quelli di una sessione di riconoscimento in cui la zona target viene inibita. La sequenza di test svolti è sinteticamente riassunta dai punti seguenti:

\begin{itemize}
	\item \textbf{Moto esplicito}
	\begin{enumerate}
		\item Rilevazione della soglia per il movimento reale: indica quanti punti mettere nella figura per avere una accuratezza del 79\% circa;
		\item Test su movimento esplicito circolare: test che dà come output l'accuratezza del partecipante;
		\item Test su movimento esplicito circolare con impulsi TSM: test con impulsi TMS rilasciati per ogni presentazione dei pattern, l'output indica l'accuratezza del partecipante.
	\end{enumerate}

	\item \textbf{Moto implicito}
	\begin{enumerate}
		\item Rilevazione della soglia per il movimento implicito: indica quante coppie di punti mettere nella figura per avere una accuratezza del 79\% circa;
		\item Test su movimento implicito:  test che dà come output l'accuratezza del partecipante;
		\item Test su movimento implicito con impulsi TSM:  test con impulsi TMS rilasciati per ogni presentazione dei pattern, l'output indica l'accuratezza del partecipante.
	\end{enumerate}
\end{itemize}

\section{Tesi}

I principali obiettivi della ricerca sono i seguenti:

\begin{enumerate}
	\item \textbf{Replicare l'esperimento di riferimento}: tale obiettivo viene perseguito mediante la realizzazione di una rete neurale convoluzionale;
	\item \textbf{Verificare se la rete proposta sia in grado di generalizzare il concetto di moto}: tale obiettivo viene perseguito allenando la rete su di una data tipologia di moto (i.e. implicito o esplicito) per poi testare le sue performance sulla tipologia opposta (e.g. allenare su moto implicito e testare su moto esplicito);
	\item \textbf{Verificare l'eventuale specializzazione degli strati della rete nel riconoscimento del moto}: tale obiettivo viene perseguito testando la rete molteplici volte, disattivando ogni volta uno strato diverso, e valutando le performance di predizione. 
\end{enumerate}