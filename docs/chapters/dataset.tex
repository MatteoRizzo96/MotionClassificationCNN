% !TEX encoding = UTF-8
% !TEX TS-program = pdflatex
% !TEX root = ../main.tex

\chapter{Dataset}

Nel seguente capitolo descriviamo il dataset utilizzato e le modalità con cui viene generato.

\section{Immagini utilizzate nell'esperimento}

Il dataset utilizzato nei vari esperimenti è consistito in una raccolta di immagini, generate attraverso script appositi che emulano il più fedelmente possibile le sequenze di immagini visualizzate durante l'esperimento condotto sui soggetti partecipanti al test reale.

\subsection{Rappresentazione della serie di immagini}
\label{ssec:3d2ddim}

Durante l'esperimento sono mostrate al soggetto partecipante due sequenze di immagini, ciascuna ripetuta due volte, e gli viene chiesto di indicare quale delle due contenga moto implicito. Nel nostro caso, la rete neurale da noi utilizzata agisce su singole immagini, e pertanto abbiamo utilizzato due metodologie diverse per rappresentare una \textit{successione di immagini}, che per nostra scelta implementativa è sempre composta da 4 frame. \\
Nel primo caso abbiamo semplicemente generato 4 immagini distinte che verranno fornite alla rete come immagini separate. Per riferirci a immagini generate con questo tipo di codifica, nella quale ognuna rappresenta un frame, abbiamo adottato il nome di immagini "2D". Generando un dataset di $n$ immagini bidimensionali saranno generate complessivamente $4\cdot n$ di frame, quattro per ogni immagine. \\ 
Come codifica alternativa, più semplice da gestire in alcuni casi, abbiamo generato delle immagini "3D", nelle quali alle due dimensioni (altezza e larghezza) è aggiunta la dimensione temporale utilizzando la \textit{bit depth} che permette di compattare i 4 frame in un'unica immagine con 4 canali di colore, ciascuno identificante un frame.\\
Provando a visualizzare un'immagine "3D" con un visualizzatore di immagini si vedrà pertanto un'immagine a colori, nella quale ognuno dei 4 colori rappresenta una diversa traslazione e movimento circolare a partire dall'immagine iniziale (figura \ref{fig:2d3d}).

\begin{figure}
	\centering
	\subfigure[Immagine \textit{2D}: un frame di una serie temporale (uno dei quattro)]{\includegraphics[width=.48\textwidth]{dataset/cm_2d}}
	\hfill
	\subfigure[Immagine \textit{3D}: tutta la serie temporale compattata in un'immagine a 4 canali]{\includegraphics[width=.48\textwidth]{dataset/cm_3d}}
	\hfill
	\caption{Esempi di immagini 2D e 3D di movimento circolare con coerenza 200}
	\label{fig:2d3d}
\end{figure}

\subsection{Tipologie di immagini utilizzate}

Le immagini generate possono essere classificate in quattro tipologie:
\begin{itemize}
	\item \textit{Dipoli circolari}: queste immagini sono generate \red{NON NE HO IDEA}. Una serie di immagini di questo tipo è mostrata in figura \ref{fig:seriedipolic};
	\item \textit{Movimento circolare}: una sequenza di immagini è costituita da una serie di immagini ruotate rispetto all'immagine iniziale. Un esempio è mostrato in figura \ref{fig:seriemovcirc};
	\item \textit{Dipoli traslazionali}: \red{NON SO COSA SIA UN DIPOLO}. Un esempio è mostrato in figura \ref{fig:seriedipolit}; 
	\item \textit{Movimento traslazionale}: una sequenza di immagini viene generata traslando nella direzione individuata da un certo angolo. Tale angolo può essere specificato e permette di ottenere traslazioni orizzontali o verticali. Una serie di esempio è mostrato in figura \ref{fig:seriemovtrasl}, con angolo di 45\textdegree.
\end{itemize}

\begin{figure}
	\centering
	\subfigure[Frame 1]{\includegraphics[width=.48\textwidth]{dataset/cd0}}
	\hfill
	\subfigure[Frame 2]{\includegraphics[width=.48\textwidth]{dataset/cd1}}
	\hfill
	\subfigure[Frame 3]{\includegraphics[width=.48\textwidth]{dataset/cd2}}
	\hfill
	\subfigure[Frame 4]{\includegraphics[width=.48\textwidth]{dataset/cd3}}
	\hfill
	\caption{Una serie di immagini di tipo \textit{dipoli circolari} [coerenza=120]}
	\label{fig:seriedipolic}
\end{figure}

\begin{figure}
	\centering
	\subfigure[Frame 1]{\includegraphics[width=.48\textwidth]{dataset/cm0}}
	\hfill
	\subfigure[Frame 2]{\includegraphics[width=.48\textwidth]{dataset/cm1}}
	\hfill
	\subfigure[Frame 3]{\includegraphics[width=.48\textwidth]{dataset/cm2}}
	\hfill
	\subfigure[Frame 4]{\includegraphics[width=.48\textwidth]{dataset/cm3}}
	\hfill
	\caption{Una serie di immagini di tipo \textit{movimento circolare} [coerenza=120]}
	\label{fig:seriemovcirc}
\end{figure}

\begin{figure}
	\centering
	\subfigure[Frame 1]{\includegraphics[width=.48\textwidth]{dataset/td0}}
	\hfill
	\subfigure[Frame 2]{\includegraphics[width=.48\textwidth]{dataset/td1}}
	\hfill
	\subfigure[Frame 3]{\includegraphics[width=.48\textwidth]{dataset/td2}}
	\hfill
	\subfigure[Frame 4]{\includegraphics[width=.48\textwidth]{dataset/td3}}
	\hfill
	\caption{Una serie di immagini di tipo \textit{dipoli traslazionali} \newline [coerenza=120, theta=45]}
	\label{fig:seriedipolit}
\end{figure}

\begin{figure}
	\centering
	\subfigure[Frame 1]{\includegraphics[width=.48\textwidth]{dataset/tm0}}
	\hfill
	\subfigure[Frame 2]{\includegraphics[width=.48\textwidth]{dataset/tm1}}
	\hfill
	\subfigure[Frame 3]{\includegraphics[width=.48\textwidth]{dataset/tm2}}
	\hfill
	\subfigure[Frame 4]{\includegraphics[width=.48\textwidth]{dataset/tm3}}
	\hfill
	\caption{Una serie di immagini di tipo \textit{movimento traslazionale} \newline [coerenza=120, theta=45]}
	\label{fig:seriemovtrasl}
\end{figure}

\newpage
\subsection{Valore di coerenza}

Per tutte le tipologie di immagini, viene definita \textit{coherency o coerenza} il numero di punti nell'immagine che sono coerenti con il movimento generato. Tutti gli altri punti saranno "rumore", ovvero non seguiranno alcuno schema specifico. Più alto è il valore di punti coerenti e più è facile visualizzare il moto, esplicito o implicito. Un'immagine con coerenza a 0 avrà tutti i punti generati casualmente (figura \ref{fig:coherencyex}).

\begin{figure}[ht]
	\subfigure[Frame 1 di serie a coerenza 0 (solo rumore)]{\includegraphics[width=.48\textwidth]{dataset/cohe_0_cm_1}}
	\hfill
	\subfigure[Frame 2 di serie a coerenza 0 (solo rumore)]{\includegraphics[width=.48\textwidth]{dataset/cohe_0_cm_2}}
	\hfill
	\subfigure[Frame 1 di serie con coerenza 120]{\includegraphics[width=.48\textwidth]{dataset/cd2}}
	\hfill
	\subfigure[Frame 2 di serie con coerenza 120]{\includegraphics[width=.48\textwidth]{dataset/cd3}}
	\hfill
	\caption{Esempi di immagini di dipoli circolari con coerenza diversa}
	\label{fig:coherencyex}
\end{figure}


\section{Generazione del dataset}

La generazione dei dataset può essere avviata eseguendo in python il file \\\texttt{generate\_dataset.py}. All'interno del file \texttt{params.json} è possibile specificare quali dataset generare. Di seguito è riportato un esempio di tale file. 

\begin{lstlisting}[language=json,firstnumber=1, label={list:jsondataset}, caption={Esempio di configurazione del \textit{json} per la generazione di due dataset},captionpos=b]
{
  "general": {
    "scale": 2,
    "blur": false
  },
  "datasets": [
    {
      "n_images": [400, 200],
      "image_type": ["circular_motion", "translational_dipoles"],
      "dimension": "2D",
      "coherency": 120,
      "generate": true
    },
    {
      "n_images": 300,
      "image_type": "circular_motion",
      "dimension": "3D",
      "coherency": 80,
      "generate": true
    }
  ]
}
\end{lstlisting}

Avviando lo script vengono generate due collezioni di immagini, una di immagini coerenti e una di immagini con solo rumore.\\
È possibile creare dataset misti, ovvero contenenti diversi tipi di immagini (e.g movimento traslazionale, dipoli circolari, ecc), specificando parametri per ogni tipologia di immagini. \texttt{image\_type} può assumere i seguenti valori, corripondenti ai 4 tipi di immagini descritte in precedenza:
\begin{itemize}
	\item \texttt{circular\_motion};
	\item \texttt{circular\_dipoles};
	\item \texttt{translational\_motion};
	\item \texttt{translational\_dipoles}.
\end{itemize}

L'angolo di traslazione è indicato al parametro \texttt{theta} e può essere omesso nel caso di generazione di dataset circolari, per il quale non è rilevante.\\ La \texttt{dimensione} delle immagini può essere "3D" o "2D", con il significato già spiegato nel paragrafo precedente (§ \ref{ssec:3d2ddim}). \\ Il parametro generale \texttt{blur} indica se si vuole generare dataset con blurring gaussiano o meno, mentre \texttt{scale} controlla la dimensione dell'immagine una volta salvata su file.\\
Nel caso in un dataset misto si voglia lasciare un parametro uguale per ogni tipo di immagine è possibile scrivere un singolo parametro, invece che riportare una lista di parametri identici. È però necessario indicare, in questo caso, almeno la tipologia di immagini come una lista.\\

Per avviare la generazione del dataset è quindi sufficiente configurare il file json con i parametri desiderati ed eseguire lo script \texttt{generate\_dataset.py}. I vari insiemi di immagini saranno generati in formato \textit{jpg}, in cartelle separate all'interno della cartella \texttt{dataset/}.\\
Ad esempio per il primo dataset nell'esempio \ref{list:jsondataset} sono generate due cartelle all'interno di \texttt{dataset/mixed\_dataset\_cm\_td\_2D\_120/}. Nella cartella \texttt{coherent/} sono inserite le $400 \cdot 4 + 200 \cdot 4 = 2400$ immagini, fra dipoli circolari e traslazionali. Nella cartella \texttt{noise/} è generato il medesimo numero di immagini di entrambi i tipi, ma contenenti solo rumore.

\subsection{Struttura della cartella \texttt{scripts}}

Tutti le funzioni per la generazione del dataset sono raccolte all'interno della cartella \texttt{scripts}. Di seguito è mostrata la struttura di tale cartella. \\

\begin{folder}[Struttura della cartella \texttt{scripts}]
	\mbox{}\\
	\dirtree{%
		.1 scripts/.
		.2 dataset\_generator/.
		.3 generate\_dataset.py\DTcomment{Main file}.
		.3 image\_builders.py\DTcomment{Funzioni legate all'implementazione}.
		.3 image\_writers.py.
		.3 params.json\DTcomment{Specifiche dei/del dataset da generare}.		
		.2 gp\_generator/.
		.3 gp\_circular\_motion.py\DTcomment{Script per il movimento circolare}.
		.3 gp\_dipoles.py\DTcomment{Script per i dipoli circolari}.
		.3 gp\_generator.py.
		.3 gp\_translational\_motion.py.
		.3 gp\_translational.py.
		.2 utility/.
		.3 json\_handler.py\DTcomment{Funzioni per il parsing del file json}.
	}
\end{folder}



