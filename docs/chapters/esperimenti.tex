% !TEX encoding = UTF-8
% !TEX TS-program = pdflatex
% !TEX root = ../main.tex

\chapter{Esperimenti}

\section{Piano sperimentale}

In questa sezione vengono elencati gli esperimenti condotti per sostenere le tesi esposte in §1.4 "Tesi". Ciascun esperimento viene condotto con tre soglie di coerenza dei punti nelle immagini che costituiscono il dataset: 60-90-120. \\

Ciascun esperimento è identificato univocamente da un codice strutturato nel modo seguente:

\begin{center}
	[Modello][Coerenza][Categoria][ID]
\end{center}

Dove:
\begin{itemize}
	\item \textbf{Modello}: indica la tipologia di rete utilizzata per effettuare l'esperimento. Può assumere i seguenti valori:
	\begin{itemize}
		\item \textbf{2D}: rete con convoluzione bidimensionale;
		\item \textbf{3D}: rete con convoluzione tridimensionale.
	\end{itemize}
	\item \textbf{Coerenza}: indica la soglia di coerenza nelle immagini del dataset (60, 90 o 120);
	\item \textbf{Categoria}: indica la categoria del'esperimento. Può assumere i seguenti valori:
	\begin{itemize}
	\item \textbf{P}: esperimento preliminare;
	\item \textbf{M}: esperimento su riconoscimento del moto;
	\item \textbf{S}: esperimento su specializzazione.
	\end{itemize}
	\item \textbf{ID}: è un codice numerico incrementale.
\end{itemize}

\subsection{Esperimenti preliminari}

Esperimenti finalizzari alla valutazione preliminare delle metriche prodotte da training e test sullo stesso tipo di dataset.

\subsubsection{Modello 2D}

\begin{table}[H]
	\rowcolors{1}{mylightgray}{white}
	\begin{tabularx}{\textwidth}{|c|X|c|c|}
		\hline 
		\textbf{Codice} &
		\textbf{Descrizione} \\ 
		\hline
		2DP0 &
		Training e test su movimento implicito circolare \\ 
		\hline
		2DP1 &
		Training e test su movimento implicito traslazionale \\ 
		\hline
		2DP2 & 
		Training e test su movimento implicito traslazionale e circolare \\ 
		\hline 
	\end{tabularx}
	\caption{Esperimenti preliminari con modello 2D}
	\label{esperimenti-preliminari-2D}
\end{table}

\newpage

\subsubsection{Modello 3D}

\begin{table}[H]
	\rowcolors{1}{mylightgray}{white}
	\begin{tabularx}{\textwidth}{|c|X|}
		\hline 
		\textbf{Codice} &
		\textbf{Descrizione} \\ 
		
		\hline 3DP0 &
		Training e test su movimento implicito circolare \\ 
		
		\hline 3DP1 &
		Training e test su movimento implicito traslazionale \\ 
		
		\hline 3DP2 & 
		Training e test su movimento implicito traslazionale e circolare \\
		
		\hline 3DP3 &
		Training e test su movimento esplicito circolare \\ 
		
		\hline 3DP4 &
		Training e test su movimento esplicito traslazionale \\ 
		
		\hline 3DP5 & 
		Training e test su movimento esplicito traslazionale e circolare \\ 
		
		\hline 3DP6 &
		Training e test su movimento implicito e esplicito circolare \\ 
		
		\hline 3DP7 &
		Training e test su movimento implicito e esplicito traslazionale \\ 
		
		\hline 3DP8 & 
		Training e test su movimento implicito e esplicito traslazionale e circolare \\ 
		\hline 
	\end{tabularx}
	\caption{Esperimenti preliminari con modello 3D}
	\label{esperimenti-preliminari-3D}
\end{table}

\subsection{Esperimenti su riconoscimento moto}

Esperimenti che mirano a determinare se la rete riesca o meno a generalizzare il concetto di moto.

\subsubsection{Modello 2D}

\begin{table}[H]
	\rowcolors{1}{mylightgray}{white}
	\begin{tabularx}{\textwidth}{|c|X|}
		\hline 
		\textbf{Codice} &
		\textbf{Descrizione} \\ 
		
		\hline 2DM0 &
		Training su movimento implicito traslazionale e test su movimento implicito circolare \\ 
		
		\hline 2DM1 &
		Training su movimento implicito circolare e test su movimento implicito traslazionale \\ 
		
		\hline 2DM2 & 
		Training su movimento implicito traslazionale e circolare e test su movimento implicito traslazionale \\ 
		
		\hline 2DM3 & 
		Training su movimento implicito traslazionale e circolare e test su movimento implicito circolare \\ 
		\hline 
	\end{tabularx}
	\caption{Esperimenti su riconoscimento moto con modello 2D}
	\label{esperimenti-riconoscimento-2D}
\end{table}

\newpage

\subsubsection{Modello 3D}

\begin{table}[H]
	\rowcolors{1}{mylightgray}{white}
	\begin{tabularx}{\textwidth}{|c|X|}
		\hline 
		\textbf{Codice} &
		\textbf{Descrizione} \\ 
		
		\hline 3DM0 &
		Training su movimento implicito traslazionale e test su movimento esplicito traslazionale \\ 
		
		\hline 3DM1 &
		Training su movimento esplicito traslazionale e test su movimento implicito traslazionale \\ 
		
		\hline 3DM2 & 
		Training su movimento implicito circolare e test su movimento esplicito circolare \\ 
		
		\hline 3DM3 & 
		Training su movimento esplicito circolare e test su movimento implicito circolare \\ 
		
		\hline 3DM4 & 
		Training su movimento implicito traslazionale e circolare e test su movimento esplicito traslazionale e circolare \\ 
		
		\hline 3DM5 & 
		Training su movimento esplicito traslazionale e circolare e test su movimento implicito traslazionale e circolare \\ 
		
		\hline 3DM6 & 
		Training su movimento implicito e esplicito traslazionale e test su movimento implicito e esplicito circolare \\
		
		\hline 3DM7 & 
		Training su movimento implicito e esplicito circolare e test su movimento implicito e esplicito traslazionale \\ 
		
		\hline 3DM8 & 
		Training su movimento implicito e esplicito traslazionale e circolare e test su movimento implicito e esplicito traslazionale e circolare \\ 
		\hline 
	\end{tabularx}
	\caption{Esperimenti su riconoscimento moto con modello 3D}
	\label{esperimenti-riconoscimento-3D}
\end{table}

\subsection{Esperimenti su specializzazione}

Esperimenti che mirano ad evidenziare se la rete specializzi o meno alcuni strati nel riconoscimento di un determinato tipo di moto, per ogni modello (2D e 3D) e per ogni soglia di coerenza (60-90-120).

\subsubsection{Modello 2D}

\begin{table}[H]
	\rowcolors{1}{mylightgray}{white}
	\begin{tabularx}{\textwidth}{|c|X|}
		\hline 
		\textbf{Codice} &
		\textbf{Descrizione} \\ 
		
		\hline 2DS0 &
		Training e test su movimento implicito circolare rimuovendo uno strato alla volta \\ 
		
		\hline 2DS1 &
		Training e test su movimento implicito traslazionale rimuovendo uno strato alla volta \\ 
		
		\hline 2DS2 & 
		Training e test su movimento implicito traslazionale e circolare rimuovendo uno strato alla volta \\ 
		\hline 
	\end{tabularx}
	\caption{Esperimenti su specializzazione con modello 2D}
	\label{esperimenti-specializzazione-2D}
\end{table}

\newpage

\subsubsection{Modello 3D}

\begin{table}[H]
	\rowcolors{1}{mylightgray}{white}
	\begin{tabularx}{\textwidth}{|c|X|}
		\hline 
		\textbf{Codice} &
		\textbf{Descrizione} \\ 
		
		\hline 3DS0 &
		Training e test su movimento implicito circolare rimuovendo uno strato alla volta \\ 
		
		\hline 3DS1 &
		Training e test su movimento implicito traslazionale rimuovendo uno strato alla volta \\ 
		
		\hline 3DS2 & 
		Training e test su movimento implicito traslazionale e circolare rimuovendo uno strato alla volta \\
		
		\hline 3DS3 &
		Training e test su movimento esplicito circolare rimuovendo uno strato alla volta \\ 
		
		\hline 3DS4 &
		Training e test su movimento esplicito traslazionale rimuovendo uno strato alla volta \\ 
		
		\hline 3DS5 & 
		Training e test su movimento esplicito traslazionale e circolare rimuovendo uno strato alla volta \\ 
		
		\hline 3DS6 &
		Training e test su movimento implicito e esplicito circolare rimuovendo uno strato alla volta \\ 
		
		\hline 3DS7 &
		Training e test su movimento implicito e esplicito traslazionale rimuovendo uno strato alla volta \\ 
		
		\hline 3DS8 & 
		Training e test su movimento implicito e esplicito traslazionale e circolare rimuovendo uno strato alla volta \\ 
		\hline 
	\end{tabularx}
	\caption{Esperimenti su specializzazione con modello 3D}
	\label{esperimenti-specializzazione-3D}
\end{table}

\section{Risultati}

\subsection{Esperimenti preliminari}

Esperimenti finalizzari alla valutazione preliminare delle metriche prodotte da training e test sullo stesso tipo di dataset.

\subsubsection{Modello 2D}

\begin{table}[H]
	\rowcolors{1}{mylightgray}{white}
	\begin{tabularx}{\textwidth}{|c|X|c|c|}
		\hline 
		\textbf{Codice} &
		\multicolumn{2}{c}{Training} \\ Loss & Accuracy & \\
		\multicolumn{2}{c}{Test} \\ Loss & Accuracy \\
		\hline
		2DP0 &
		Training e test su movimento implicito circolare \\ 
		\hline
		2DP1 &
		Training e test su movimento implicito traslazionale \\ 
		\hline
		2DP2 & 
		Training e test su movimento implicito traslazionale e circolare \\ 
		\hline 
	\end{tabularx}
	\caption{Esperimenti preliminari con modello 2D}
	\label{esperimenti-preliminari-2D}
\end{table}

\newpage

\subsubsection{Modello 3D}

\begin{table}[H]
	\rowcolors{1}{mylightgray}{white}
	\begin{tabularx}{\textwidth}{|c|X|}
		\hline 
		\textbf{Codice} &
		\textbf{Descrizione} \\ 
		
		\hline 3DP0 &
		Training e test su movimento implicito circolare \\ 
		
		\hline 3DP1 &
		Training e test su movimento implicito traslazionale \\ 
		
		\hline 3DP2 & 
		Training e test su movimento implicito traslazionale e circolare \\
		
		\hline 3DP3 &
		Training e test su movimento esplicito circolare \\ 
		
		\hline 3DP4 &
		Training e test su movimento esplicito traslazionale \\ 
		
		\hline 3DP5 & 
		Training e test su movimento esplicito traslazionale e circolare \\ 
		
		\hline 3DP6 &
		Training e test su movimento implicito e esplicito circolare \\ 
		
		\hline 3DP7 &
		Training e test su movimento implicito e esplicito traslazionale \\ 
		
		\hline 3DP8 & 
		Training e test su movimento implicito e esplicito traslazionale e circolare \\ 
		\hline 
	\end{tabularx}
	\caption{Esperimenti preliminari con modello 3D}
	\label{esperimenti-preliminari-3D}
\end{table}

\subsection{Esperimenti su riconoscimento moto}

Esperimenti che mirano a determinare se la rete riesca o meno a generalizzare il concetto di moto.

\subsubsection{Modello 2D}

\begin{table}[H]
	\rowcolors{1}{mylightgray}{white}
	\begin{tabularx}{\textwidth}{|c|X|}
		\hline 
		\textbf{Codice} &
		\textbf{Descrizione} \\ 
		
		\hline 2DM0 &
		Training su movimento implicito traslazionale e test su movimento implicito circolare \\ 
		
		\hline 2DM1 &
		Training su movimento implicito circolare e test su movimento implicito traslazionale \\ 
		
		\hline 2DM2 & 
		Training su movimento implicito traslazionale e circolare e test su movimento implicito traslazionale \\ 
		
		\hline 2DM3 & 
		Training su movimento implicito traslazionale e circolare e test su movimento implicito circolare \\ 
		\hline 
	\end{tabularx}
	\caption{Esperimenti su riconoscimento moto con modello 2D}
	\label{esperimenti-riconoscimento-2D}
\end{table}

\newpage

\subsubsection{Modello 3D}

\begin{table}[H]
	\rowcolors{1}{mylightgray}{white}
	\begin{tabularx}{\textwidth}{|c|X|}
		\hline 
		\textbf{Codice} &
		\textbf{Descrizione} \\ 
		
		\hline 3DM0 &
		Training su movimento implicito traslazionale e test su movimento esplicito traslazionale \\ 
		
		\hline 3DM1 &
		Training su movimento esplicito traslazionale e test su movimento implicito traslazionale \\ 
		
		\hline 3DM2 & 
		Training su movimento implicito circolare e test su movimento esplicito circolare \\ 
		
		\hline 3DM3 & 
		Training su movimento esplicito circolare e test su movimento implicito circolare \\ 
		
		\hline 3DM4 & 
		Training su movimento implicito traslazionale e circolare e test su movimento esplicito traslazionale e circolare \\ 
		
		\hline 3DM5 & 
		Training su movimento esplicito traslazionale e circolare e test su movimento implicito traslazionale e circolare \\ 
		
		\hline 3DM6 & 
		Training su movimento implicito e esplicito traslazionale e test su movimento implicito e esplicito circolare \\
		
		\hline 3DM7 & 
		Training su movimento implicito e esplicito circolare e test su movimento implicito e esplicito traslazionale \\ 
		
		\hline 3DM8 & 
		Training su movimento implicito e esplicito traslazionale e circolare e test su movimento implicito e esplicito traslazionale e circolare \\ 
		\hline 
	\end{tabularx}
	\caption{Esperimenti su riconoscimento moto con modello 3D}
	\label{esperimenti-riconoscimento-3D}
\end{table}

\subsection{Esperimenti su specializzazione}

Esperimenti che mirano ad evidenziare se la rete specializzi o meno alcuni strati nel riconoscimento di un determinato tipo di moto, per ogni modello (2D e 3D) e per ogni soglia di coerenza (60-90-120).

\subsubsection{Modello 2D}

\begin{table}[H]
	\rowcolors{1}{mylightgray}{white}
	\begin{tabularx}{\textwidth}{|c|X|}
		\hline 
		\textbf{Codice} &
		\textbf{Descrizione} \\ 
		
		\hline 2DS0 &
		Training e test su movimento implicito circolare rimuovendo uno strato alla volta \\ 
		
		\hline 2DS1 &
		Training e test su movimento implicito traslazionale rimuovendo uno strato alla volta \\ 
		
		\hline 2DS2 & 
		Training e test su movimento implicito traslazionale e circolare rimuovendo uno strato alla volta \\ 
		\hline 
	\end{tabularx}
	\caption{Esperimenti su specializzazione con modello 2D}
	\label{esperimenti-specializzazione-2D}
\end{table}

\newpage

\subsubsection{Modello 3D}

\begin{table}[H]
	\rowcolors{1}{mylightgray}{white}
	\begin{tabularx}{\textwidth}{|c|X|}
		\hline 
		\textbf{Codice} &
		\textbf{Descrizione} \\ 
		
		\hline 3DS0 &
		Training e test su movimento implicito circolare rimuovendo uno strato alla volta \\ 
		
		\hline 3DS1 &
		Training e test su movimento implicito traslazionale rimuovendo uno strato alla volta \\ 
		
		\hline 3DS2 & 
		Training e test su movimento implicito traslazionale e circolare rimuovendo uno strato alla volta \\
		
		\hline 3DS3 &
		Training e test su movimento esplicito circolare rimuovendo uno strato alla volta \\ 
		
		\hline 3DS4 &
		Training e test su movimento esplicito traslazionale rimuovendo uno strato alla volta \\ 
		
		\hline 3DS5 & 
		Training e test su movimento esplicito traslazionale e circolare rimuovendo uno strato alla volta \\ 
		
		\hline 3DS6 &
		Training e test su movimento implicito e esplicito circolare rimuovendo uno strato alla volta \\ 
		
		\hline 3DS7 &
		Training e test su movimento implicito e esplicito traslazionale rimuovendo uno strato alla volta \\ 
		
		\hline 3DS8 & 
		Training e test su movimento implicito e esplicito traslazionale e circolare rimuovendo uno strato alla volta \\ 
		\hline 
	\end{tabularx}
	\caption{Esperimenti su specializzazione con modello 3D}
	\label{esperimenti-specializzazione-3D}
\end{table}
